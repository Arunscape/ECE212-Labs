%preamble
\documentclass[letterpaper]{article}
\synctex=1

\usepackage{listings}
\lstset{
language=assembly
breaklines=true,
%frame=single,
xleftmargin=0
}

%actual document
\begin{document}

  %titlepage
  \begin{titlepage}
    \begin{center}
        %%stuff
    \end{center}
\end{titlepage}

%table of tableofcontents
\tableofcontents
\newpage

\section{Introduction}
  %Intro

\section{Design}

  \subsection{Part A}
    part a

  \subsection{Part B}
    partb


\section{Testing}
  \subsection{Part A}
    part a

  \subsection{Part B}
    partb

\section{Questions}

    \subsection{Question 1}
      \item ``What happens when there is no exit code ‘0x0D’ provided in the initialization process? Would it cause a problem? Why or why not?''
      answer goes here

    \subsection{Question 2}
      \item ``How can our code be modified to provide a variable address range? For example, what if I only wanted to convert the first 10 data entires? ''
      answer goes here


\section{Conclusion}
  conclusions

\section{Appendix}
  \subsection{Part A Assembler Code}

    \lstinputlisting{parta.s}

  \subsection{Part B Assembler Code}

    \lstinputlisting{partb.s}

\end{document}
