%preamble
\documentclass[letterpaper]{article}
\synctex=1

\usepackage{listings}
\lstset{
language=assembly
breaklines=true,
%frame=single,
xleftmargin=0
}

%actual document
\begin{document}

  %titlepage
  \begin{titlepage}
    \begin{center}
        %%stuff
    \end{center}
\end{titlepage}

%table of tableofcontents
\tableofcontents
\newpage

\section{Introduction}
  %Intro

\section{Design}

  \subsection{Part A}
    part a

  \subsection{Part B}
    partb


\section{Testing}
  \subsection{Part A}
    part a

  \subsection{Part B}
    partb

\section{Questions}

    \subsection{Question 1}
      \item ``What happens when there is no exit code ‘0x0D’ provided in the initialization process? Would it cause a problem? Why or why not?''
      answer goes here

    \subsection{Question 2}
      \item ``How can our code be modified to provide a variable address range? For example, what if I only wanted to convert the first 10 data entires? ''
      answer goes here


\section{Conclusion}
  conclusions

\section{Appendix}
  \subsection{Part A Assembler Code}

    %\lstinputlisting{parta.s}
    /* DO NOT MODIFY THIS --------------------------------------------*/
    .text

    .global AssemblyProgram

    AssemblyProgram:
    lea      -40(%a7),%a7 /*Backing up data and address registers */
    movem.l %d2-%d7/%a2-%a5,(%a7)
    /*----------------------------------------------------------------*/

    /******************************************************************/
    /* General Information ********************************************/
    /* File Name: Lab1a.s *********************************************/
    /* Names of Students: Arun Woosaree and Navras Kamal             **/
    /* Date: 1/29/2018                                               **/
    /* General Description:                                          **/
    /*                                                               **/
    /******************************************************************/

    /*Write your program here******************************************/

    movea.l #0x2300000, %a1 	/* save input address to a1*/
    movea.l #0x2310000, %a2 	/* save output address to a2*/

    /* let a value in quotation marks be the ASCII value of the character enclosed by the quotation marks*/

    loop: 						/* the looping function*/
     move.l (%a1), %d2 			/* move the value at address a1 to d2, call this 'inval' from henceforth*/

     cmp.l #0x0D, %d2			/* Check if the inval is the enter code*/
     beq end					/* if it is, go to the end of the program (breaking the loop)*/

     cmp.l #0x2F, %d2			/* compare inval to the hex value of "0"*/
     blt err					/* if inval is less than ASCII zero it is not valid, throw an error*/

     cmp.l #0x3A, %d2 			/* compare the inval to the hex value of ":", which is one ASCII value higher than "9"*/
     blt zeronine				/* if it is less than the value of ":" then it must be a value between "0" and "9"*/
     							/* 	   thus go to the proper part of the code to handle this value*/

     cmp.l #0x41, %d2 			/* compare the inval to "A"*/
     blt err					/* if it is less than the "A" than it is invalid, throw an error*/

     cmp.l #0x47, %d2 			/* compare the inval to "G"*/
     blt bigathruf				/* if it is less than the value of "G" then it must be in the range "A" through "F"*/
     							/*	 thus go to the part of the code to handle these values*/

     cmp.l #0x61, %d2 			/* compare the inval to "a"*/
     blt err					/* if it is in this range it is invalid, thus throw an error*/

     cmp.l #0x67, %d2 			/* compare the inval to "g"*/
     blt littleathruf			/* if it is less than "g" then it must be in the range "a" through "F"*/
     							/*	 thus go to the part of the code to handle these values*/

    err:						/* if the inval is equal to or above "g" then the code will naturally continue here*/
     move.l #0xFFFFFFFF, (%a2) 	/* throw the error code to the output address location*/
     bra endloop				/* go to the end of the loop before restarting the loop*/

    zeronine:					/* inval is between "0" and "9"*/
     sub.l #0x30, %d2 			/* subtract the hex value of "0" from inval, which will leave a value from 0x0 to 0x9, for "0" to "9" respectively*/
     move.l %d2, (%a2) 			/* move this calculted hex value to the output address location*/
     bra endloop				/* go to the end of the loop before restarting the loop*/

    bigathruf:					/* inval is between "A" and "F"*/
     sub.l #0x41, %d2   			/* subtracts the hex value of "A" d2. This is the difference between d2 and the character and "A"*/
     add.l #0xA, %d2  			/* adds the value of "A" to d2, which will make it into the hex representation of the original ASCII value*/
     move.l %d2, (%a2)			/* move this value to the output address location*/
     bra endloop				/* go to the end of the loop before restarting the loop*/

    littleathruf:				/* inval is between "a" and "f"*/
     sub.l #0x61, %d2   			/* subtracts the hex value of "a" d2. This is the difference between d2 and the character and "a"*/
     add.l #0xA, %d2  			/* adds the value of "a" to d2, which will make it into the hex representation of the original ASCII value*/
     move.l %d2, (%a2)			/* move this value to the output address location*/
     bra endloop				/* go to the end of the loop before restarting the loop*/

    endloop:					/* handles code to be executed before the start of a new loop*/
     add.l #0x4, %a1 			/* increment the input address by 4*/
     add.l #0x4, %a2 			/* increment the output address by 4*/
     bra loop					/* restart the loop*/

    end:						/* end the custom part of the program*/

    /*End of program **************************************************/

    /* DO NOT MODIFY THIS --------------------------------------------*/
    movem.l (%a7),%d2-%d7/%a2-%a5 /*Restore data and address registers */
    lea      40(%a7),%a7
    rts
    /*----------------------------------------------------------------*/




  \subsection{Part B Assembler Code}
    \begin{lstlisting}
    /* DO NOT MODIFY THIS --------------------------------------------*/
    .text

    .global AssemblyProgram

    AssemblyProgram:
    lea      -40(%a7),%a7 /*Backing up data and address registers */
    movem.l %d2-%d7/%a2-%a5,(%a7)
    /*----------------------------------------------------------------*/

    /******************************************************************/
    /* General Information ********************************************/
    /* File Name: Lab1a.s *********************************************/
    /* Names of Students: Arun Woosaree and Navras Kamal             **/
    /* Date: 1/29/2018                                               **/
    /* General Description:                                          **/
    /*                                                               **/
    /******************************************************************/

    /*Write your program here******************************************/

    movea.l #0x2300000, %a1 	/* save input address to a1*/
    movea.l #0x2320000, %a2 	/* save output address to a2*/

    /* let a value in quotation marks be the ASCII value of the character enclosed by the quotation marks*/

    loop: 						/* the looping function*/
     move.l (%a1), %d2 			/* move the value at address a1 to d2, call this 'inval' from henceforth*/

     cmp.l #0x0D, %d2			/* Check if the inval is the enter code*/
     beq end					/* if it is, go to the end of the program (breaking the loop)*/

     cmp.l #0x41, %d2 			/* compare the inval to "A"*/
     blt err					/* if it is less than the "A" than it is invalid, throw an error*/

     cmp.l #0x5B, %d2 			/* compare the inval to "["*/
     blt bigathruz				/* if it is less than the value of "[" then it must be in the range "A" through "Z"*/
                  /*	 thus go to the part of the code to handle these values*/

     cmp.l #0x61, %d2 			/* compare the inval to "a"*/
     blt err					/* if it is in this range it is invalid, thus throw an error*/

     cmp.l #0x7B, %d2 			/* compare the inval to "{"*/
     blt littleathruz			/* if it is less than "{" then it must be in the range "a" through "z"*/
                  /*	 thus go to the part of the code to handle these values*/

     bigathruz:					/* inval is between "A" and "Z"*/
     add.l #0x20, %d2			/* adds the hex difference between "A" and "a", making it into the lowercase equivalent*/
     move.l %d2, (%a2)			/* move this value to the output address location*/
     bra endloop				/* go to the end of the loop before restarting the loop*/
     /*TODO*/
    littleathruz:				/* inval is between "a" and "z"*/
     sub.l #0x20, %d2			/* subtracts the hex difference between "a" and "A", making it into the uppercase equivalent*/
     move.l %d2, (%a2)			/* move this value to the output address location*/
     bra endloop				/* go to the end of the loop before restarting the loop*/
     /*TODO*/
     err:						/* if the inval is not a valid character then the code will naturally continue here*/
     move.l #0xFFFFFFFF, (%a2) 	/* throw the error code to the output address location*/
     bra endloop				/* go to the end of the loop before restarting the loop*/

     endloop:					/* handles code to be executed before the start of a new loop*/
     add.l #0x4, %a1 			/* increment the input address by 4*/
     add.l #0x4, %a2 			/* increment the output address by 4*/
     bra loop					/* restart the loop*/

     end:


    /*End of program **************************************************/

    /* DO NOT MODIFY THIS --------------------------------------------*/
    movem.l (%a7),%d2-%d7/%a2-%a5 /*Restore data and address registers */
    lea      40(%a7),%a7
    rts
    /*----------------------------------------------------------------*/

    \end{lstlisting}

\end{document}
