%preamble
\documentclass[letterpaper]{article}
\synctex=1

\usepackage{listings}
\lstset{
%language=Assembler
% breaklines=true
%frame=single,
%xleftmargin=-1pt
}
\usepackage{geometry}
\usepackage{array}
\usepackage{lipsum}

\usepackage{graphicx}
\usepackage{float}
\graphicspath{ {images/} }
\usepackage{siunitx}
\usepackage{float}
% \usepackage[hidelinks]{hyperref}
% \usepackage[linktoc=all,hidelinks]{hyperref}
% \usepackage{bookmark}
% \usepackage[section]{placeins}
%
% \newenvironment{changemargin}[2]{%
% \begin{list}{}{%
% \setlength{\topsep}{0pt}%
% \setlength{\leftmargin}{#1}%
% \setlength{\rightmargin}{#2}%
% \setlength{\listparindent}{\parindent}%
% \setlength{\itemindent}{\parindent}%
% \setlength{\parsep}{\parskip}%
% }%
% \item[]}{\end{list}}

% \usepackage{tabu}
% \usepackage{blindtext}
%actual document
\begin{document}

  %titlepage
  \begin{titlepage}
    \begin{center}

      \LARGE
      ECE 212 Lab - Introduction to Microprocessors

      Department of Electrical and Computer Engineering

      University of Alberta

      \vspace{2cm}

      Lab 2: Introduction to Addressing Modes

      \vspace{5cm}
      \Large

      \begin{tabular}{ | m{5cm} | m{5cm} | }
        \hline
        Student Name & Student \\
        \hline
        Arun Woosaree & XXXXXXX \\
        \hline
        Navras Kamal & 1505463 \\
        \hline
      \end{tabular}

    \end{center}
\end{titlepage}

%table of tableofcontents

\tableofcontents

% \vfill
\newpage

\section{Introduction}
  The purpose of this lab is to learn and apply multiple methods of indirect
  register addressing using the ColdFire Assembly language in a hands on
  environment. This will go over the methods of Register Indirect with Offeset,
  Indexed Register Indirect and Postincrement Register Indirect.

  Register Indirect with Offset is written in the form of (offset, \%address), where
  the offset is a constant value. This will cause the (address + offset) memory
  location to be accessed instead of the basic addressed location. This operation does
  not modify the value stored in the address register. Indexed Register Indirect is
  written as (\%offset, \%address) where the offset is a data register. Similar to
  Register Indirect with Offset this does not modify the address register and points
  to location (offset + address), but in this case the offset is a data register, thus
  the offset can vary depending on the contents of the data register. This is useful
  to reduce redundancy or to handle loops of unknown or variable number of iterations.
  Finally, there is Postincrement Register Indirect, which accesses the location at
  (address), then afterwards will increment address by the size of the operand.
  Therefore, a longword operation will increase address by 4, a word by 2 and a byte by 1.
  This is more useful when iterating over an array of unknown or variable size, when you
  do not need to preserve the value of the original address, and when you are handling
  values of constant size.

  This lab will be split into two sections. The first part will entail the
  demonstration of the above three methods of indirect registry accessing.  This
  will be done with a simple example in which there is a pair of arrays of
  values. The nth element of the first array will need to be added with the nth
  element of the second array, and their sum stored in the nth element of a third
  array designated for outputting the values.  This process will be completed a
  total of three times, once with each of the forms of registry accessing above.

  The second part of the lab will involve an example in which the use of these
  indirect registering will be tested. In this case it will be demonstrated by
  performing a Trapezoidal Approximation of a curve. A Trapezoidal Approximation
  is a way of estimating the area under a curve, or the Integral of the function
  represented by the curve. This example will focus solely on the two dimensional
  case. It will take an array of points in the x axis, and a second array of
  points in the y axis. These values are formed into Cartesian Coordinates by
  pairing up the nth term of the first array with the $n^{th}$term of the second,
  creating an (x,y) coordinate pair. This represents the $n^{th}$ point on the graph.

  These can be used alongside the formula for the area under a trapezoid, which is
  $A= h\cdot(\frac{a+b}{2})$. In this formula it is assuming a trapezoid with parallel
  top and bottom of any length (a and b respectively), and a distance between them of
  h.  It takes the average between the two parallel lengths and multiplies them by the
  distance between them to find the total area enclosed.  In our example, this equation
  and the definitions of the variables will need to be modified somewhat. Instead, we
  will be rotating the system $\ang{90}$, such that h does not represent the height, but
  rather the horizontal distance along the x axis between the two parallel lines.
  So the values of a and b will instead be
  two different y coordinates on the graph, and h will be the absolute value of the
  difference between their respective x coordinates, or
  $\Delta x$. By assuming that the x coordinate
  of b is greater than that of a it is possible to rewrite the equation in terms of two
  different Cartesian coordinates representing two distinct places on the graph. For our
  calculations it will be easier to divide at the end of the overall calculation. Thus,
  the new equation is $A=\frac{(x_1-x_0)\cdot(y_0+y_1)}{2}$. Using this formula it will be
  possible to gain an approximation of the area under the graph, assuming that the
  distance in the x axis between the points is kept small.

  Thus, by using the above formula it can be seen how it is possible to use a series of
  cartesian coordinates to calculate an approximation of the integral of a two
  dimensional function. These trapezoidal calculations will need to be added together
  then output to a specific memory location, and in order to iterate through the lists
  of x and y values, indirect registering is the most effective way to handle the values.
  This will demonstrate an understanding of the use of Register Indirect methods.

  Many of the concepts in this lab build off of ideas from the previous two, including the
  use and debugging of the ColdFire system from lab 0 and the application of functions such
  as loops, as explored in lab 1.


\section{Design}

\subsection{Part A: Different Addressing Modes}
For part A of this lab, we wrote code to add the contents of two arrays
stored at different memory locations, and the result is stored at another
specified memory location. This is done using three  different addressing modes.
In the first part, we use Register Indirect With Offset, in the second part, we
use Indexed Register Indirect, and in the third part, we use Postincrement
Register Indirect. All information is stored as a long word (32 bits). The
operational code is provided at memory location 0x2300000, which contains the
information for where to access the arrays in memory and where to store the sum.
For example, the number stored at 0x2300000 contains the size of the arrays. The
operational code is stored as follows:

    \begin{enumerate}
      \item 0x2300000 - Size of arrays
      \item 0x2300004 - Address of first array
      \item 0x2300008 - Address of second array
      \item 0x230000C - Address of where to store the sum with Register Indirect With Offset
      \item 0x2300010 - Address of where to store the sum with Indexed Register Indirect
      \item 0x2300014 - Address of where to store the sum with Postincrement Register Indirect
    \end{enumerate}

    \begin{figure}[h!]
      \centering
      \includegraphics[width=.6\textwidth]{mema.jpg}
      \caption{An illustration of how the data is organized in part A}
    \end{figure}

    \noindent The \textit{SetZeros.s} and the \textit{DataStorage.s}
    files, which were provided, were used to initialize memory contents.

    \subsubsection{Part 1: Register Indirect With Offset}
    We chose address register a1 to store the contents at 0x230000C, which is
    where the address to output the sum is located. This was done by loading the
    effective address into a1, and moving the contents of the memory location
    stored at a1 into a1 itself. In a similar fashion, address register a2 was
    chosen to store the contents at 0x2300004, which stores the starting address
    of the first array to be added. Similarly, address register a3 was chosen to
    store the location of where the second array to be added starts, and that
    memory address is stored at 0x2300008. To add the first two numbers in the
    arrays, the contents of the memory location pointed to by a2 were moved into
    data register d3, and the contents of the memory location pointed to by a3
    were added to d3. The result in d3 was then stored at the memory location
    pointed to by a1. Next, these three lines of code were copy-pasted and
    modified to add offsets to the memory locations.  So, to add the next two
    numbers in each array, the contents of the memory location (4,\%a2) was
    moved to d3, the contents of the memory location (4,\%a3) was added to d3,
    and the result in d3 was moved to the memory location (4,\%a1). The same
    method was applied to add the third numbers in the arrays, but this time
    with an offset of 8.


    \subsubsection{Part 2: Indexed Register Indirect}
    The loading of the memory addresses in part 2 was done very similarly to part 1.
    Address registers a2 and a3 still contain the memory locations of where the
    first and second array begin. However, the contents at memory location
    0x2300010 was moved to a1, which is where the result is stored for this
    part. Data register d1 was chosen to store the amount by which we want to
    offset, and d2 was chosen to be our counter variable for a loop.
    Additionally, both these data registers were cleared. Inside the loop, the
    contents of the memory location stored by a2 were moved to d3, just like in
    part 1. Except this time, it was offset by the number in d1. So, on the
    first loop iteration, it would be displaced by 0. Similarly, the contents of
    the memory location pointed to by a3, displaced by the value at d1, is added
    to d3. The result stored in d3 is then moved to the memory location a1, also
    displaced by the number stored in d1. At the end of the loop, we increment
    the offset d1 by 4, and the counter is incremented by 1. Then, we compare our
    counter to the size of the arrays, which is a value stored at 0x2300000, and
    if our counter is less than the size of the array, we continue the loop.


    \subsubsection{Part 3: Postincrement Register Indirect}
    Loading of the memory addresses in part 3 was also very similar to parts 1 and
    2. Address registers a2 and a3 are unchanged, but the  contents at memory
    location 0x2300014 was moved to address register a1, which is where the
    result is stored in this part. We chose data register d2 to be our counter
    variable, which was initially cleared with the value of 0. Just like for
    part 2, we have a loop, but this time there are no offsets. Instead, we move
    the contents of the memory location pointed to by a2 to d3, and
    postincrement a2. Similarly, the contents fo the memory location pointed to
    by a3 is moved to d3, and a3 is post incremented. At the end of the loop,
    the counter is incremented by 1, and if it is less than the size of the
    arrays (stored at 0x2300000), the loop continues.


    % \subsubsection{Part A Sample Calculations of Conversion}
    %     \textbf{from the last lab}
    %     input = `9' = 0x39\\
    %     0x39 - 0x30 = 0x9\\
    %     \\
    %     input = `E' = 0x45\\
    %     0x45 - 0x41 = 0x4\\
    %     0x4 + 0xA = 0xE\\
    %     \\
    %     input = `d' = 0x64\\
    %     0x64 - 0x61 = 0x3\\
    %     0x3 + 0xA = 0xD



  \subsection{Part B: Trapezoidal Rule}

    In part B of this lab,we write a program that calculates the area
    underneath a curve ($y=f(x)$), using the trapezoidal rule:
    $$ \int_a^b f(x) dx \approx \sum_{k=1}^{N} \frac{f(x_{k-1})+f(x_k)}{2}\Delta x_k$$
    The x and y data points for the curve are stored in arrays, at different
    memory addresses. For convenience, the distance between each X data point is either one
    or 2 units. Similar to part A, we have some operational code stored at 0x2300000, which contains the
    information for where to access the arrays in memory and where to store the output.
    The operational code is as follows:

    \begin{enumerate}
      \item 0x2300000 - Number of Data Points
      \item 0x2300004 - Address of where the X data points are stored
      \item 0x2300008 - Address of where the Y data points are stored
      \item 0x230000C - Address of temporary storage space
      \item 0x2300010 - Address of where to store the final output
    \end{enumerate}

    \begin{figure}[h!]
      \centering
      \includegraphics[width=0.6\textwidth]{memb.jpg}
      \caption{An illustration of how the data is organized in part B}
    \end{figure}

    In our initialization, we chose d0 to store the total number of data points,
    by moving the contents at 0x2300000 to d0, then subtracting 1 from d0 to
    avoid an off-by-one indexing error. a0 was chosen to store the address of
    the array with X data points, so the contents at 0x2300004 were moved to a0.
    Similarly, the contents at 0x2300010 were moved to a2, to store the address
    of the array with Y data points. Data register d1 was used as a counter, and
    d7 stores the total area. Both of these data registers were cleared with
    initial value 0. Initial data points $x_0$ and $y_0$ were stored in d2 and
    d3, respectively. Once our initialization is done, we enter a main loop,
    where we decided to do our checks for exiting the loop at the beginning. The
    loop exits if our counter (d0) is equal to the number of points (d1),
    otherwise the counter is incremented. In the loop, the contents of the
    memory address pointed to by a0 is moved to d4, and post incremented, so
    that d4 holds $x_k$. Similarly, the contents of the memory address pointed
    to by a1 is moved to d5, and post incremented, so that d5 holds $f(x_k)$. We
    get $f(x_{k-1})+f(x_k)$ by adding d5 to d3 and storing the result in d3.
    Then, $\Delta x$ is calculated by subtracting d4 from d2, storing the
    negation of that result in d2. If $\Delta x$ was 2, we multiplied d3, which
    was $f(x_{k-1})+f(x_k)$ by 2 and then added the result to d7. If $\Delta x$
    was not 2, we just added $f(x_{k-1})+f(x_k)$ to d7. At the end of the loop,
    $x_k$ and $f(x_k)$ from the current iteration are set to be $x_{k-1}$
    $f(x_{k-1})$ respectively, for the next iteration by moving the contents of
    d4 to d2 and also moving the contents of d5 to d3. When the loop terminates,
    we check if the number in d7 is odd by checking its least significant bit.
    If it is odd, we add one to it. Finally, the number in d7 is divided by 2
    using a right bit shift operation, and the result is moved into the memory
    location pointed to by a2, which is where the output should go. We check if
    the number in d7 is odd before dividing it by 2, so that our program would
    round the number up if we get a fraction, as per the assignment
    specifications.

%     \subsubsection{Part B Sample Calculations of Conversion}
%     \textbf{from the last lab}
%     input = `M' = 0x4D\\
%     0x4D + 0x20 = 0x6D\\
%     0x6D = 'm'\\
%     \\
%     input = `d' = 0x64\\
%     0x64 - 0x20 = 0x44\\
%     0x44 = 'D'


\section{Testing}
  \subsection{Part A}
    Initially, we visually tested our code by using the debugger in Eclipse IDE.
    While stepping through the code, we would check the values at relevant
    memory locations, and the data and address registers. Initially, for part 3,
    we had an error where the sum calculated in the output array was all zeros.
    However, we later discovered that while exiting our loop in part 2, we
    jumped to the end of the program, so part 3 would never run. When the bugs
    were ironed out, After we fixed this minor bug, we went on to the next phase
    of testing. Our code was tested using the provided \textit{Lab2aTest.s}
    file. More specifically, this program was moved into the project folder,
    downloaded to the ColdFire microcontroller, and the MTTTY serial monitor was
    loaded to monitor the expected output. Our code was further tested by
    replacing the `DataStorage.s' file with the other variants provided named:
    \textit{DataStorage1.s}, \textit{DataStorage2.s}, and
    \textit{DataStorage3.s}. Finally, our program, which produced the correct
    output, was verified by a lab TA, who further tested our code by modifying
    the DataStorage file and verifying the output.
    \begin{figure}[H]
      \centering
      \includegraphics[width=.65\textwidth]{part1.jpg}
      \caption{MTTTY output when testing our Part A solution}
    \end{figure}

  \subsection{Part B}
    Although the parts were different, the procedure for testing our code for
    part B was similar to the process described above in Part A. We visually
    inspected our code in the Eclipse IDE, used the Eclipse debugger to step
    through our code, and monitored relevant memory addresses and registers. We
    initially got a huge number, above 63 million, which we correctly identified
    as an off-by-one indexing error. This problem was rectified by subtracting
    one from d0, which is what we used to store the total number of points. We
    also included some code towards the end which  would make sure our code
    rounds up as per the lab specifications if we got a fraction instead of a
    whole number when calculating the area (as verified by a lab TA). However,
    this code seemed to have no effect, since our program produced the correct
    output anyways. Our program still worked with this code, however as it was not
    fully tested we left it commented out in our submission. After review we have
    seen that there is a more elegant way to reach the same solution using bit tests
    but we have left it as it was at the time that it was reviewed by the TAs in lab.
    In any case, we used the provided files \textit{Lab2bTest.s}, and the
    \textit{Datastorage5.s} files to verify our solution by downloading the
    program to the ColdFire microcontroller, and monitoring the output in MTTTY.
    Finally, our solution was verified by a lab TA. As per the lab instructions,
    we used \textit{DataStorage5.s} and \textit{SetZeros5.s} files, although we
    did further test our program using \textit{DataStorage4.s} with a lab TA and
    found no issues.


    % \begin{figure}[H]
    %   \centering
    %   \includegraphics[width=0.6\textwidth]{part2ds4.jpg}
    %   \caption{MTTTY output when testing our Part B solution with \textit{Datastorage4.s}}
    % \end{figure}

    \begin{figure}[H]
      \centering
      \includegraphics[width=0.9\textwidth]{part2ds5.jpg}
      \caption{MTTTY output when testing our Part B solution with \textit{DataStorage5.s}}
    \end{figure}

\section{Questions}
  \begin{enumerate}
    \item \textit{What are the advantages of using the different addressing modes covered in this lab?}
          % \textbf{A: }
          \begin{enumerate}
            \item Register Indirect With Offset
              \begin{enumerate}
                \item Great if you know the position of data relative to some memory location
                \item Doesn't modify the address register's value
                \item Can access any specific memory address you want relative to the value
                      stored in the address register
              \end{enumerate}
            \item Indexed Register Indirect
            \begin{enumerate}
              \item Doesn't modify the address register's value
              \item You can offset in terms of a variable, so you have greater control over the offset
              \item Works well in loops in order to access multiple sequential memory locations
                    while being able to easily return to the original address as the value in the
                    address register is never changed
            \end{enumerate}
            \item Postincrement Register Indirect
            \begin{enumerate}
              \item You don't have to keep track of the offset, it automatically
                    increments the address register by the size of the data
                    you're working with
              \item Very useful for dealing with arrays, as in part B
              \item Works nicely in loops to access multiple sequential memory
                    locations and be pointing to the next memory location afterwards
            \end{enumerate}
          \end{enumerate}

    \item \textit{If the differece between the X data points are not restricted to be either one
                  or two units, how would you modify your program to calculate the area? You do
                  not need to do this in your code.}\\
          \textbf{A: }
          Instead of hard coding two scenarios where $\Delta x$ is either 1 or 2,
          we could just divide $f(x_{k-1})+f(x_k)$ by 2 with a bit shift operation
          to the right by one at each iteration rather than at the end, and then use the \textit{muls.l}
          instruction to multiply the result by $\Delta x$ to match the equation for the trapezoidal rule.
    \item \textit{From the data points, what is the function (y=f(x))? What is the percent error between the
                  theoretical calculated area and the one obtain in your program?}\\
          % \textbf{A: }
          \\From the data points, we can tell the function is $y=x^2$.\\
          The value obtained in our program was \textbf{170710}.\\
          The actual value should be:\\
          $$\int_0^{80} x^2 dx = \frac{x^3}{3}\Big|_0^{80} = \frac{80^3}{3} = \frac{512000}{3}= \textbf{170666.666...} $$
          Therefore, the percent error is:
          $$\frac{|\frac{512000}{3}-170710|}{\frac{512000}{3}}\times 100 = \frac{13}{512}\% = \textbf{0.025390625\%}$$

  \end{enumerate}

\section{Conclusion}
The first part of this lab demonstrated how to use multiple forms of accessing
registers indirectly and their individual benefits. The forms of indirect access
explored in this lab were Register Indirect with Offset, Indexed Register
Indirect and Postincrement Register Indirect. The second part of this lab was a
practical demonstration which could be solved using the aforementioned methods.
In this example a graph is stored as (x,y) pairs, with each half of the pair
being an element in one of two different arrays. The issues we experienced were
mostly user error, such as accidentally skipping over portions of the code,
however in testing we ended up having to reflash the board relatively frequently,
which slowed the pace of testing somewhat. However, once we got the feel for the
testing and with some assistance from the TAs we were able to resolve our issues.
Overall the lab went smoothly, and has indeed succeeded at it's goal of furthering
our knowledge of the various methods of referencing memory locations.

\newpage
\section{Appendix}
  %\textwidth=600pt
  \subsection{Part A Assembler Code}
    % \begin{changemargin}{-2cm}{-2cm}
    \lstinputlisting{code/Lab2a.s}
  % \end{changemargin}
\newpage


  \subsection{Part A Flowchart Diagrams}

  \subsubsection{Part 1: Register Indirect With Offset}
  \vspace{2cm}
  \noindent\makebox[\textwidth]{\includegraphics[width=\paperwidth,height=.5\paperheight,keepaspectratio]{partaflowchart1.jpg}}
\newpage

  \subsubsection{Part 2: Indexed Register Indirect}
  \vspace{2cm}
  \noindent\makebox[\textwidth]{\includegraphics[width=\paperwidth,height=.5\paperheight,keepaspectratio]{partaflowchart2.jpg}}
\newpage

  \subsubsection{Part 3: Postincrement Register Indirect}
  \vspace{2cm}
  \noindent\makebox[\textwidth]{\includegraphics[width=\paperwidth,height=.5\paperheight,keepaspectratio]{partaflowchart3.jpg}}
\newpage


  \subsection{Part B Assembler Code}
    \lstinputlisting{code/Lab2b.s}
\newpage

  \subsection{Part B Flowchart Diagram}
  \vspace{2cm}
    \noindent\makebox[\textwidth]{\includegraphics[width=\paperwidth,height=.6\paperheight,keepaspectratio]{partbflowchart.jpg}}
\newpage
% \@
% % \pagenumbering{gobble}
% % \addcontentsline{toc}{section}{Marking Sheet}
% % \section{Marking Sheet}
% % \clearpage
% \addtocounter{section}{1}
% % \addcontentsline{toc}{section}{Marking Sheet}
% \addcontentsline{toc}{section}{\protect\numberline{\thesection} Marking Sheet}

\section{Marking Sheet}
\end{document}
